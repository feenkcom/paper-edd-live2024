% Modeled after sample-sigplan.tex
\documentclass[sigplan,anonymous,review,10pt]{acmart}
%\documentclass[sigplan,10pt]{acmart}
%% \BibTeX command to typeset BibTeX logo in the docs
\AtBeginDocument{%
  \providecommand\BibTeX{{%
    Bib\TeX}}}
\setcopyright{acmlicensed}
\copyrightyear{2024}
\acmYear{2024}
\acmDOI{XXXXXXX.XXXXXXX}
%% These commands are for a PROCEEDINGS abstract or paper.
\acmConference[LIVE 2024]{LIVE 2024}{Oct.\ 20-25, 2024}{Pasadena, CA}
\acmISBN{978-1-4503-XXXX-X/18/06}
% ============================================================
\input{macros}
\input{st80.tex}
% ============================================================
% Macros for this paper
\newcommand*{\smallimg}[1]{%
    \raisebox{-.3\baselineskip}{%
        \includegraphics[
        height=\baselineskip,
        width=\baselineskip,
        keepaspectratio,
        ]{#1}%
    }%
}
%\renewcommand{\nbc}[3]{} % To hide reviewer comments
\newcommand\on[1]{\nbc{ON}{#1}{olive}} % add more author macros here
\newcommand\tg[1]{\nbc{TG}{#1}{blue}}
\newcommand\ac[1]{\nbc{AC}{#1}{teal}}
%\newcommand\steve[1]{\nbc{Steven}{#1}{red}} % Costiou
%\newcommand\ab[1]{\nbc{Alex}{#1}{violet}} % Bergel
%\newcommand\tk[1]{\nbc{Timo}{#1}{brown}} % Kehrer
\usepackage{caption}
\captionsetup{aboveskip=5pt,belowskip=-10pt} % Adjust the space around figure captions
%\usepackage{enumitem}
%\setlist[description]{font=\itshape}
\newcommand{\GT}{\lst{GT}\xspace} % In case we want to display it differently ...
%\newcommand\lmaf{\lst{Ludo\-Move\-Assert\-ion\-Fail\-ure}\xspace}
% ============================================================
% Optionally anonymize selected names
\newboolean{anonymous}
\setboolean{anonymous}{true}
\newcommand\anonymize[2]{\ifthenelse{\boolean{anonymous}}{#2}{#1}\xspace}
\newcommand\feenk{\anonymize{feenk}{anonymous company}}
\newcommand\deet{{\tt deet}\xspace}
% ============================================================
\begin{document}
%% The "title" command has an optional parameter,
%% allowing the author to define a "short title" to be used in page headers.
\title{Example-driven development: \\ bridging tests and documentation}
\author{Andrei Chi\c{s}}
\affiliation{%
  \institution{feenk gmbh}
  \city{Wabern}
  \country{Switzerland}}
\email{andrei.chis@feenk.com}
\author{Tudor G\^irba}
\affiliation{%
  \institution{feenk gmbh}
  \city{Wabern}
  \country{Switzerland}}
\email{tudor.girba@feenk.com}
\author{Oscar Nierstrasz}
\affiliation{%
  \institution{feenk gmbh}
  \city{Wabern}
  \country{Switzerland}}
\email{oscar.nierstrasz@feenk.com}

\renewcommand{\shortauthors}{Chi\c{s} et al.}

\begin{abstract}
Software systems should be \emph{explainable}, that is, they should help us to answer questions while exploring, developing or using them.
Textual documentation is a very weak form of explanation, since it is not causally connected to the code, so easily gets out of date.
\emph{Tests}, on the other hand, are causally connected to code, but they are also a weak form of explanation.
Although some tests encode interesting scenarios that answer certain questions about how the system works, most tests tend to be uninteresting.

\emph{Examples} are tests that are also factories for interesting system entities.
Instead of simply succeeding or failing, an example returns the object under test so that it can be inspected, or reused to compose further tests.
An example \emph{is} causally connected to the system, is always live and tested, and can be embedded into live documentation.
Although technically examples constitute just a tiny modification to test methods, their impact is potentially ground breaking.
We show
\begin{inparaenum}[(i)]
	\item how EDD enriches TDD with live programming,
	\item how examples can be molded with tiny tools to answer analysis questions, and	\item how examples can be embedded in live documentation to make a system explainable.
\end{inparaenum}
\end{abstract}

%\keywords{TODO.}

%\received{20 February 2007}
%\received[revised]{12 March 2009}
%\received[accepted]{5 June 2009}

\maketitle

% NB: Max 6 pages

% ============================================================
\section{Background: Examples = Tests + Factories}\label{sec:intro}

Unit tests, as originally introduced by Beck \cite{Beck94c}, exercise objects, known as \emph{fixtures}, and evaluate assertions over these objects.
% Beck94c Simple Smalltalk Testing: With Patterns
% See https://en.m.wikipedia.org/wiki/SUnit
% https://web.archive.org/web/20150315073817/http://www.xprogramming.com/testfram.htm
Fixtures are created with the help of shared \emph{setup} methods, and, once the test method succeeds or fails, are simply discarded.
Only if a test fails do we have access to the fixture, from within the debugger.

Gaelli noted that there was a missed opportunity here, and proposed a modified approach to unit testing in which tests return their fixtures, that is, they serve as factories for \emph{examples}~\cite{Gael06b}.
% Gael06b Modeling Examples to Test and Understand Software (PhD)
The output of a test --- an example --- can then be (re-)used as the input (fixture) for another test.
Examples can then be composed to form higher-level scenarios~\cite{Gael07a}.
% Gael07a Composing Tests from Examples (JOT)

In principle, any XUnit testing framework, for language X, can be extended to support examples.
JExample\footnote{\url{https://scg.unibe.ch/research/jexample}} extends JUnit 4, allowing test methods not only to return examples, but also to accept as input one or more other examples from the same test example class.
% https://scg.unibe.ch/research/jexample
Interestingly, refactoring tests as examples establishes a partial order amongst test methods.
The impact of this is that
\begin{inparaenum}[(i)]
	\item code duplication is reduced as common preambles to complex tests are refactored as shared examples, and
	\item defect localization is improved since fewer tests will fail~\cite{Kuhn08a}.
\end{inparaenum}
% Kuhn08a JExample: Exploiting Dependencies Between Tests to Improve Defect Localization
H{\"a}nsenberger showed that tests in the wild often contain much duplicated code.
By performing dynamic analysis of tests, one can largely automate the process of migrating classical unit tests to examples~\cite{Haen08b,Haen09a}.
% Haen08b Using Dynamic Analysis for API Migration (PCODA)
% Haen09a Defect Isolation As Responsibility of the Framework --- Automated API Migration from JUnit to JExample (MSc)

Although examples are already interesting as a means to make explicit the otherwise implicit dependencies between tests to reduce duplicated code and improve defect localization, it turns out that examples can impact the software process in other, important ways.
In the remainder of this paper, we will show how examples
\begin{inparaenum}[(i)]
	\item enrich the TDD process with live programming opportunities,
	\item enable the \emph{moldable development} of tiny analysis tools, and
	\item offer a means to augment documentation with live, interactive examples, as a step towards making software systems explainable.
\end{inparaenum}

% ============================================================
\section{EDD: TDD + Examples}\label{sec:edd}

A key motivation for integrating unit tests into mainstream software development was to enable continuous refactoring of evolving software systems \cite{Beck00a}.
% Beck00a, Extreme Programming Explained: Embrace Change
Test-Driven Development (TDD)~\cite{Beck03a} offers a way not only to ensure that tests are produced in tandem with the implementation of a software system, but importantly to exploit the potential for tests to serve as requirements that can drive the design and implementation process.
% Beck03a Test Driven Development: By Example

Ignoring the ideological debates that have been raging \todo{REF?} over the pros and cons of TDD and other Agile practices, we can observe the following issues with TDD.

\begin{inparaenum}[(i)]
	\item TDD advocates that one should always start development by writing a test first.
But how do you know how to express what to test?
In practice, you have to ``guess first'' to imagine an API to express the interactions needed to exercise and test functionality you want to implement.
Of course, this is how tests ``drive'' development, but could there be an easier way to get to the test code that you need to write?
	\item TDD then recommends that you write just enough code to make the test pass.
But how do you know where to start coding?
Again you have to ``guess first'' to figure out where to begin coding.
	\item Once you have a green test, what can you do with it except read the code?
A red test is useful, as long as it brings you to a debugger that lets you explore why the test has failed, but a green test is uninteresting and useless.
How could we make it useful?
\end{inparaenum}

\emph{Example-Driven Development} (EDD) offers a new take on TDD in which examples drive the development process.
It is similar to TDD, but can differ in important ways by exploiting opportunities raised by live programming.
In short,
\begin{inparaenum}[(i)]
	\item instead of starting by writing a test, we start by inspecting a bare-bones example, and incrementally writing code that will become a test scenario to produce an interesting example,
	\item instead of writing code to make a test pass, we iteratively and incrementally grow the example, and extract assertions that express what is interesting about the example,
and
	\item the green test returns the example that we can inspect, interact with, and, as we shall see, embed in live documentation.
\end{inparaenum}

Let's see how this can work with a small, running example.

\todo{Step through an EDD scenario as we did in the EDD blog.
Use the Prices use case?}
% https://lepiter.io/feenk/example-driven-development-ekmic0u0o8swpblzkcdzy608s/

% ============================================================
\section{Moldable Examples}\label{sec:moldable}

\emph{Moldable development} is an approach to constructing \emph{explainable software systems} by augmenting the objects of the software system with dozens of tiny analysis tools that can answer questions about the system.
Moldable development is made possible with the help of \emph{moldable tools}~\cite{Chis17a}, such as code browsers, debuggers, and object inspectors, that can adapt themselves to the run-time context of an application to enable these analysis tools.
% Chis17a Moldable Tools for Object-oriented Development

For example, consider the screenshot of the Ludo\footnote{\url{https://en.m.wikipedia.org/wiki/Ludo}} in figure \todo{copy from MD patterns paper}.
At the left we see in an object inspector a GUI \emph{Board} view of a running instance of the game that has terminated with player B winning.
In the middle we see a \emph{Moves} view of the same instance, showing us all the past moves leading to the concluding state of the game.
Finally, at the right we inspect a particular move, showing us how the game state was updated in move \#109.

These views have each been created with a few lines of code, in the first and last cases leveraging the existing GUI view of the Ludo game.
The object inspector recognizes that the Ludo game object is an instance of the \st{GtLudoRecordingGame} class, 
\todo{fix hyphenation}
which has been extended with several custom views defined as annotated methods of that class.
Similarly the move object is an instance of the \st{GtLudoMove} class, which has been extended with other views specific to moves.

Other custom tools include
\begin{inparaenum}[(i)]
	\item \emph{custom actions} (\eg buttons), which perform a task and possible spawn another tool such as an inspector, a code editor or an external web browser, and
	\item \emph{custom searches}, which query the running object model and spawn a tool such as an object inspector on the result.
\end{inparaenum}
\todo{others?}




    
Examples are both input and output of MD
Input: raw example
Output: molded example augmented with tools

Basic EDD lifecycle is augmented with steps to let examples tell stories about themselves
Just as examples are developed live, so are the tools

% ============================================================
\section{Explainable Systems using Examples}\label{sec:explainable}

	
		Examples can be embedded in live documentation
		Project Diaries / Knowledge Bases
			Notebook pages with live examples
		GT Book as the prime example
			Examples
			Statistics


% ============================================================
\section{Conclusion}\label{sec:conclusion}


%\begin{acks}
%\end{acks}

\bibliographystyle{ACM-Reference-Format}
\bibliography{eddBridge}


\end{document}
\endinput

\begin{inparaenum}[(i)]
	\item 
\end{inparaenum}
