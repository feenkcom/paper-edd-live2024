% Modeled after sample-sigplan.tex
\documentclass[sigplan,anonymous,review,10pt]{acmart}
%\documentclass[sigplan,10pt]{acmart}
%% \BibTeX command to typeset BibTeX logo in the docs
\AtBeginDocument{%
  \providecommand\BibTeX{{%
    Bib\TeX}}}
\setcopyright{acmlicensed}
\copyrightyear{2024}
\acmYear{2024}
\acmDOI{XXXXXXX.XXXXXXX}
%% These commands are for a PROCEEDINGS abstract or paper.
\acmConference[LIVE 2024]{LIVE 2024}{Oct.\ 20-25, 2024}{Pasadena, CA}
\acmISBN{978-1-4503-XXXX-X/18/06}
% ============================================================
\input{macros}
\input{st80.tex}
\usepackage{subfigure}
% ============================================================
% Macros for this paper
\newcommand*{\smallimg}[1]{%
    \raisebox{-.3\baselineskip}{%
        \includegraphics[
        height=\baselineskip,
        width=\baselineskip,
        keepaspectratio,
        ]{#1}%
    }%
}
%\renewcommand{\nbc}[3]{} % To hide reviewer comments
\newcommand\on[1]{\nbc{ON}{#1}{olive}} % add more author macros here
\newcommand\tg[1]{\nbc{TG}{#1}{blue}}
\newcommand\ac[1]{\nbc{AC}{#1}{teal}}
%\newcommand\steve[1]{\nbc{Steven}{#1}{red}} % Costiou
%\newcommand\ab[1]{\nbc{Alex}{#1}{violet}} % Bergel
%\newcommand\tk[1]{\nbc{Timo}{#1}{brown}} % Kehrer
\usepackage{caption}
\captionsetup{aboveskip=5pt,belowskip=-10pt} % Adjust the space around figure captions
%\usepackage{enumitem}
%\setlist[description]{font=\itshape}
\newcommand{\GT}{\lst{GT}\xspace} % In case we want to display it differently ...
%\newcommand\lmaf{\lst{Ludo\-Move\-Assert\-ion\-Fail\-ure}\xspace}
% ============================================================
% Optionally anonymize selected names
\newboolean{anonymous}
\setboolean{anonymous}{true}
\newcommand\anonymize[2]{\ifthenelse{\boolean{anonymous}}{#2}{#1}\xspace}
\newcommand\feenk{\anonymize{feenk}{anonymous company}}
\newcommand\deet{{\tt deet}\xspace}
% ============================================================
\begin{document}
%% The "title" command has an optional parameter,
%% allowing the author to define a "short title" to be used in page headers.
\title[Example-driven development: bridging tests and documentation]{Example-driven development: \\ bridging tests and documentation}
\author{Andrei Chi\c{s}}
\affiliation{%
  \institution{feenk gmbh}
  \city{Wabern}
  \country{Switzerland}}
\email{andrei.chis@feenk.com}
\author{Tudor G\^irba}
\affiliation{%
  \institution{feenk gmbh}
  \city{Wabern}
  \country{Switzerland}}
\email{tudor.girba@feenk.com}
\author{Oscar Nierstrasz}
\affiliation{%
  \institution{feenk gmbh}
  \city{Wabern}
  \country{Switzerland}}
\email{oscar.nierstrasz@feenk.com}

\renewcommand{\shortauthors}{Chi\c{s} et al.}

\begin{abstract}
Software systems should be \emph{explainable}, that is, they should help us to answer questions while exploring, developing or using them.
Textual documentation is a very weak form of explanation, since it is not causally connected to the code, so easily gets out of date.
\emph{Tests}, on the other hand, are causally connected to code, but they are also a weak form of explanation.
Although some tests encode interesting scenarios that answer certain questions about how the system works, most tests tend to be uninteresting.

\emph{Examples} are tests that are also factories for interesting system entities.
Instead of simply succeeding or failing, an example returns the object under test so that it can be inspected, or reused to compose further tests.
An example \emph{is} causally connected to the system, is always live and tested, and can be embedded into live documentation.
Although technically examples constitute just a tiny modification to test methods, their impact is potentially ground breaking.
We show
\begin{inparaenum}[(i)]
	\item how Example-Driven Development (EDD) enriches TDD with live programming,
	\item how examples can be \emph{molded} with tiny tools to answer analysis questions, and
    \item how examples can be embedded within live documentation to make a system explainable.
\end{inparaenum}
\end{abstract}

%\keywords{TODO.}

%\received{20 February 2007}
%\received[revised]{12 March 2009}
%\received[accepted]{5 June 2009}

\maketitle

% NB: Max 6 pages

% ============================================================
\section{Background: Examples = Tests + Factories}\label{sec:intro}

Unit tests, as originally introduced by Beck \cite{Beck94c}, exercise objects, referred to as \emph{fixtures}, and evaluate assertions over these objects.
% Beck94c Simple Smalltalk Testing: With Patterns
% See https://en.m.wikipedia.org/wiki/SUnit
% https://web.archive.org/web/20150315073817/http://www.xprogramming.com/testfram.htm
Fixtures are created with the help of shared \emph{setup} methods, and, once the test method succeeds or fails, are simply discarded.
Only if a test fails do we have access to the fixture, from within the debugger.

Gaelli noted that there was a missed opportunity here, and proposed a modified approach to unit testing in which tests return their fixtures, that is, they serve as factories for \emph{examples}~\cite{Gael06b}.
% Gael06b Modeling Examples to Test and Understand Software (PhD)
The output of a test --- an example --- can then be (re-)used as the input (fixture) for another test.
Examples can then be composed to form higher-level scenarios~\cite{Gael07a}.
% Gael07a Composing Tests from Examples (JOT)

In principle, any XUnit testing framework, for language X, can be extended to support examples.
JExample\footnote{\url{https://scg.unibe.ch/research/jexample}} extends JUnit 4, allowing test methods not only to return examples, but also to accept as input one or more other examples from the same test example class.
% https://scg.unibe.ch/research/jexample
Interestingly, refactoring tests as examples establishes a partial order amongst test methods.
The impact of this is that
\begin{inparaenum}[(i)]
	\item code duplication is reduced since common preambles to complex tests are refactored as shared examples, and
	\item defect localization is improved since fewer tests will fail~\cite{Kuhn08a}.
\end{inparaenum}
% Kuhn08a JExample: Exploiting Dependencies Between Tests to Improve Defect Localization
H{\"a}nsenberger showed that tests in the wild often contain much duplicated code.
By performing dynamic analysis on tests, one can largely automate the process of migrating classical unit tests to examples~\cite{Haen08b,Haen09a}.
% Haen08b Using Dynamic Analysis for API Migration (PCODA)
% Haen09a Defect Isolation As Responsibility of the Framework --- Automated API Migration from JUnit to JExample (MSc)

Although examples are already interesting as a means to make explicit the otherwise implicit dependencies between tests to reduce duplicated code and improve defect localization, it turns out that examples can impact the software process in other, important ways.
In the remainder of this paper, we will show how examples
\begin{inparaenum}[(i)]
	\item enrich the TDD process with live programming opportunities,
	\item enable the \emph{moldable development} of tiny analysis tools, and
	\item offer a means to augment documentation with live, interactive examples, as a step towards making software systems explainable.
\end{inparaenum}

% ============================================================
\section{EDD: TDD + Examples}\label{sec:edd}

A key motivation for integrating unit tests into mainstream software development was to enable continuous refactoring of evolving software systems \cite{Beck00a}.
% Beck00a, Extreme Programming Explained: Embrace Change
Test-Driven Development (TDD)~\cite{Beck03a} offers a way not only to ensure that tests are produced in tandem with the implementation of a software system, but importantly to exploit the potential for tests to serve as requirements that can drive the design and implementation process.
% Beck03a Test Driven Development: By Example

Ignoring ongoing ideological debates over the pros and cons of TDD and other Agile practices, we can observe the following issues with TDD:
\begin{inparaenum}[(i)]
	\item TDD advocates that one should always start development by writing a test first.
But how do you know how to express what you want to test?
In practice, you have to ``guess first'' to imagine an API to express the interactions needed to exercise and test functionality you want to implement.
Of course, this is how tests ``drive'' development, but could there be an easier way to get to the test code that you need to write?
	\item TDD then recommends that you write just enough code to make the test pass.
But how do you know where to start coding?
Again you have to ``guess first'' to figure out where to begin coding.
	\item Once you have a green test, what can you do with it except read the code?
A red test is useful, as long as it brings you to a debugger that lets you explore why the test has failed, but a green test is uninteresting and useless.
How could we make it useful?
\end{inparaenum}

\emph{Example-Driven Development} (EDD) offers a new take on TDD in which examples drive the development process.
It is similar to TDD, but can differ in important ways by exploiting opportunities raised by live programming.
In short,
\begin{inparaenum}[(i)]
	\item instead of starting by writing a test, we start by inspecting a bare-bones example, and incrementally write code that will become a test scenario to produce an interesting example,
	\item instead of writing code to make a test pass, we iteratively and incrementally grow the example, and extract assertions that express what is interesting about the example,
and
	\item a green test returns an example that we can inspect, interact with, and, as we shall see, embed into live documentation.
\end{inparaenum}

Let's see how this can work with a small, running example in Glamorous Toolkit\footnote{\url{https://gtoolkit.com}} (GT), an open-source moldable development environment implemented on top of the Pharo\footnote{\url{https://pharo.org}} Smalltalk  platform.

Suppose we have an existing library of classes implementing amounts of \st{Money} in various currencies, such as \st{100 euros} or \st{10 usd}.
We would now like to implement \emph{prices} for goods, where a \st{Price} may be a concrete, fixed price, or a \emph{discounted} price, where the discount may be a fixed value or a percentage.

With TDD we would probably start by specifying a test method within a dedicated \st{TestPrice} class that sets up a fixture for a concrete fixed price of, say, \st{100 euros}, and then asserts something about it.
With examples, \emph{we could do pretty much the same}, creating an example method in a \lst{PriceExamples} class that creates the same fixture, asserts something, \emph{and returns that example object}.
But how would we create an instance of the (not yet defined) \st{Price} class?
What would we like to assert about it?
What should the API of the \st{Price} class look like?

It is true that TDD forces us to ask such questions, so in this way it \emph{drives} the design process.
\emph{However}, TDD does not otherwise help us in answering these questions.
We are forced to ``guess first'' and try to specify an interface before we can implement anything.

Instead, with EDD, we can exploit the fact that we have examples to iterative and incrementally specify the test (\ie the example method) and \emph{prototype} the object it tests.
In this way, EDD helps us to make small steps towards both specifying the test example method and implementing the code that will make the test (example) green.

\begin{figure}[h]
  \includegraphics[width=\columnwidth]{edd1-ConcretePrice100Euros}
	\caption{Prototyping a raw Price object.}
  \label{fig:exampleCreationA}
\end{figure}

We can see this illustrated in \autoref{fig:exampleCreationA} where we start (leftmost pane) by writing a snippet that creates a new \lst{ConcretePrice} object with \st{100 euros} as its \st{money} value.
With the help of \emph{fixit dialogues} (not shown) we create the new class with a \st{money} slot (instance variable) and setter method.
We can then inspect the new, raw \st{ConcretePrice} (middle pane), showing that its \st{money} slot is properly initialized.
We can also click on that slot, and inspect its value (rightmost pane).

\begin{figure}[h]
  \includegraphics[width=\columnwidth]{edd2-PrototypingAsPrice}
	\caption{Prototyping a factory method.}
  \label{fig:exampleCreationB}
\end{figure}

At this point we realize that it would be nice to have a factory method to create a \st{Price} from a \st{Money} instance.
In the context of the live \st{100 euro} \st{Money} object (\autoref{fig:exampleCreationB} bottom of left pane) we prototype the code to create a price from this instance, namely:
\begin{code}
ConcretePrice new money: self; yourself.
\end{code}
Note that \st{self} is bound to the live object we are inspecting.
Evaluating this snippet and inspecting the result (right pane) gives us the \st{ConcretePrice} object that we expect.

\begin{figure}[h]
  \includegraphics[width=\columnwidth]{edd3-ExtractAsPrice}
	\caption{Extracting a factory method.}
  \label{fig:exampleCreationC}
\end{figure}

Now that we have prototyped the factory code, we can extract it as an extension method of the \st{Money} class called \st{asPrice} using an \emph{Extract method} refactoring.
We see the refactored code (\autoref{fig:exampleCreationC}, left pane) as a ``code bubble''~\cite{Brag10a} by expanding \st{asPrice}.
We also see that the refactored snippet still yields the result we want (right pane).

\begin{figure}[h]
  \includegraphics[width=\columnwidth]{edd4-MoneyAsPrice}
	\caption{Rewriting the initial snippet.}
  \label{fig:exampleCreationD}
\end{figure}

Now we can go back to our original snippet (\autoref{fig:exampleCreationD}, left pane) and rewrite it as:
\begin{code}
100 euros asPrice.
\end{code}

\begin{figure}[h]
  \includegraphics[width=\columnwidth]{edd5-ExtractingExample}
	\caption{Extracting an example method.}
  \label{fig:exampleExtractionA}
\end{figure}

Now we have a nice snippet that creates an example that interests us.
In \autoref{fig:exampleExtractionA} we apply an \emph{Extract example} refactoring to create a new \st{PriceExamples} class with a \st{hundredEuros} example  method.

\begin{figure}[h]
  \includegraphics[width=\columnwidth]{edd6-PriceExample}
	\caption{The extracted example method.}
  \label{fig:exampleExtractionB}
\end{figure}

It is simply a method with a \st{<gtExample>} annotation (analogous to Java method annotations) that flags it as an example method, and which returns the object of interest (\autoref{fig:exampleExtractionB}, left pane with code bubble).
Our example method still does not test anything, so let's prototype that too.
We would expect our \st{hundredEuros} example to be equal to another instance that is created in the same way.

\begin{figure}[h]
  \includegraphics[width=\columnwidth]{edd7-FailedTest}
	\caption{Prototyping an assertion.}
  \label{fig:exampleExtractionC}
\end{figure}

Within the context of the live example (\autoref{fig:exampleExtractionC}, left pane) we prototype the assertion that this example object (\ie \st{self}) is equal to another object created the same way.
Unfortunately this fails (right pane) because our new \st{ConcretePrice} object has not implemented equality, so equality defaults to object identity.

\begin{figure}[h]
  \includegraphics[width=\columnwidth]{edd8ExampleWithAssertions}
	\caption{Adding the assertion to the example method.}
  \label{fig:exampleExtractionD}
\end{figure}

We implement the missing method, and now update our example method (\autoref{fig:exampleExtractionD} left pane) with the new assertion.
If we evaluate this example method, it is not only green (left), but also returns an example we can explore (right).

% ============================================================
\section{Moldable Examples}\label{sec:moldable}

When we inspect our \st{hundredEuro} example (\autoref{fig:exampleExtractionD}), instead of the original ``raw'' inspector view we see a new \emph{Money} view showing the value of the concrete price.
How did this happen?
Actually there is a step missing that we will now explain.

\emph{Moldable development} is an approach to constructing \emph{explainable software systems} by augmenting the objects of the software system with dozens of tiny analysis tools that can answer questions about the system.
It can be understood as a refinement of EDD in which objects (examples) are enhanced with custom tools during the development process.

Moldable development is made possible with the help of \emph{moldable tools}~\cite{Chis17a}, such as code browsers, debuggers, and object inspectors, that can adapt themselves to the run-time context of an application to enable these analysis tools.
% Chis17a Moldable Tools for Object-oriented Development
For example, consider the screenshot of the Ludo\footnote{\url{https://en.m.wikipedia.org/wiki/Ludo}} in figure \autoref{fig:ludoViews}.
At the left we see in an object inspector a GUI \emph{Board} view of a running instance of the game that has terminated with player B winning.
In the middle we see a \emph{Moves} view of the same instance, showing us all the past moves leading to the concluding state of the game.
Finally, at the right we inspect a particular move, showing us how the game state was updated in move \#109.

\begin{figure}[h]
  \includegraphics[width=\columnwidth]{customViews}
  \caption{Custom views of a Ludo game.}
  \label{fig:ludoViews}
\end{figure}

These views have each been created with a few lines of code, in the first and last cases leveraging the existing GUI view of the Ludo game.
The object inspector recognizes that the Ludo game object is an instance of the \lst{GtLudoRecordingGame} class, 
which has been extended with several custom views defined as annotated methods of that class.
Similarly the move object is an instance of the \st{GtLudoMove} class, which has been extended with other views specific to moves.

Two other common types of custom tools are \emph{custom actions} (\eg buttons), which perform a task and possible spawn another tool such as an inspector, a code editor or an external web browser, and \emph{custom searches}, which query the running object model and spawn a tool such as an object inspector on the result.

Examples are both the input and output of moldable development.
Typically we start with a ``raw'', unenhanced example, such as we see in \autoref{fig:exampleCreationB}: the \st{aConcretePrice} inspector view shows just a basic ``raw'' view of the instance state of the example.
As we elaborate the examples in the EDD process, we mold them with custom tools that validate the requirements expressed by the examples.
The output of the process is then a molded example that not only checks assertions about its expected behavior, but also exposes that behavior through custom tools.

In \autoref{fig:ConcretePrice} we see another demo version of our \st{ConcretePrice} class called \st{GtDConcretePrice}.
It has a custom view called \emph{Overview} (left pane) showing that the price is just a fixed amount of money.
In the right pane we see the code that implements this custom view.
The details of the implementation are not important here, but please note that this method consist of just a dozen lines or so of boilerplate code.
Most custom views in GT are short, and the average size of these methods is about a dozen lines.

\begin{figure}[h]
  \includegraphics[width=\columnwidth]{md1-ConcretePrice}
  \caption{A ConcretePrice object with the code of a custom view.}
  \label{fig:ConcretePrice}
\end{figure}

In \autoref{fig:DiscountedPrice} we see a more complex example of a price that has been discounted twice, first by a fixed amount, and then by a percentage.
In this case, the \emph{Overview} explains how the price has been calculated as a composition of discounts.

\begin{figure}[h]
  \includegraphics[width=\columnwidth]{md2-DiscountedPrice}
  \caption{A DiscountedPrice with a custom view showing how it is composed.}
  \label{fig:DiscountedPrice}
\end{figure}

% ============================================================
\section{Explainable Systems using Examples}\label{sec:explainable}

%\todo{Add something about creating narratives -- the custom views allow you to create both fixed narratives as notebook pages, and dynamic narratives by navigating to new views}

Perhaps the most compelling use of examples is within live documentation.
GT includes support for knowledge bases consisting of linked notebook pages that are composed of various kinds of snippets: formatted test, images, code in various programming languages, and live examples.
An example snippet identified an example method to be evaluated, and a view to be rendered when the notebook page is loaded.

This simple feature enables the creation of various kinds of live documentation, such as live project diaries, interactive tutorials, and live API documentation.

Let's see a few examples from the GT book, the knowledge base that comes bundled with the GT download.

In \autoref{fig:Memory} we see an extract of tutorial pages explaining how to build a graphical user interface using examples of a ``Memory'' game.\footnote{\url{https://en.wikipedia.org/wiki/Concentration_(card_game)}}
The page contains a live, embedded example of a partially played Memory game with two cards turned over.

\begin{figure}[h]
  \includegraphics[width=\columnwidth]{book1-Memory}
  \caption{Part of a tutorial in implementing a ``Memory'' game.}
  \label{fig:Memory}
\end{figure}

% ON: Skip the SPL example to save space.

% \autoref{fig:SPL} explains how to implement a simple programming language using the PetitParser framework that comes bundled with GT.
% We see an embedded example of a live interpreter of the SPL language.
% At the bottom of the example snippet we can also see how the snippet was specified, by naming the example class and method (\st{SPLCaseStudyExamples >> #print3plus4}), specifying that we do not want to see the code of the example (\st{noCode=true}), and finally which custom view of the example we would like to see instead (\st{previewShow=#gtStateFor:}).

% \begin{figure}[h]
% \includegraphics[width=\columnwidth]{book2-SPL}
% \caption{Implementing a simple programming language.}
% \label{fig:SPL}
% \end{figure}
% \st{SPLCaseStudyExamples >> #print3plus4 | noCode=true | previewShow=#gtStateFor:}

Examples can also be used effectively to explain algorithms.
In \autoref{fig:Treemap} we see a notebook page that explains how a TreeMap layout algorithm works with the help of live examples.
\todo{Fix the example to show how the treemap switches orientation}
In the figure we are diving into a node of the TreeMap to explore how it is specified.

\begin{figure}[h]
  \includegraphics[width=\columnwidth]{book3-Treemap}
  \caption{Explaining a TreeMap algorithm.}
  \label{fig:Treemap}
\end{figure}

Finally, \autoref{fig:Patterns} shows how examples can be used within notebook pages that explain the moldable development process itself in terms of a number of patterns~\cite{Nier24a}.
The \emph{Moldable Development patterns} page contains a live embedded map (an example) that allows you to navigate to pages describing the individual patterns.
These pages, such as \emph{Custom Search}, in turn contain other live examples to illustrate the patterns.

\begin{figure}[h]
  \includegraphics[width=\columnwidth]{book4-MDPatterns}
  \caption{xxx.}
  \label{fig:Patterns}
\end{figure}

All of these illustrations show not only how example methods can be used productively to produce live documentation, but that by applying moldable development principles, the live examples can be easily enhanced with lightweight, custom tools that serve to make the examples, and the systems they are part of, truly explainable.

%\todo{Should we list statistics on examples, snippets, pages etc.?\\
%There are 10804 example methods averaging 13.068 LoC
%compared with
%24552 test methods averaging 9.58 LoC.\\
%The GT Book has 456 pages, of which 97 containing example snippets, for a total of 332 example snippets.
%Two pages have 22 examples.
%}

% ============================================================
\section{Related work}\label{sec:related}

%\todo{Add something on Shri's Executable Examples paper? Wren19a}
%% https://cs.brown.edu/~sk/Publications/Papers/Published/wk-examplar/

The \emph{Test Data Builder} pattern~\cite{Free09a} introduces factory methods for test fixtures, also potentially reducing code duplication in tests, but the resulting objects are only intended as inputs for tests, not their outputs.
The resulting examples are not accessible as the output of a green test.

Cucumber~\cite{Hell17a} is a software tool that supports Behavior-Driven Development through business rules specified in the Gherkin language.
These rules include the specification of ``examples'' (AKA scenarios) that illustrate business rules.
The usage of these examples, however, is restricted to the context of the business rules, and they are not intended as the outputs of tests.

Modern testing frameworks such as pytest~\cite{Okke22a}, allow tests to be parameterized by fixtures specified as separate methods.
Here too, however, fixtures are only seen as inputs to tests, not as outputs.

Bush was the first to dream of a computerized ``memex'' of stored knowledge~\cite{Bush45a}, through which a user could trace an associative ``trail'' of interconnected text and multimedia resources.
The memex vision inspired Engelbart's NLS~\cite{Enge68a}, the first system to demonstrate human interaction with a computer mouse, windows, and hypertext features.
Knuth first pioneered the implementation of a computational notebook, called WEB, to support ``literate programming'' through the combination of text, graphics, and live code~\cite{Knut97a}.

Modern notebook systems such as Jupyter\footnote{\url{https://jupyter.org}}, MATLAB Live Scripts\footnote{\url{https://www.mathworks.com/}} and Wolfram Notebooks\footnote{\url{https://www.wolfram.com/notebooks/}} all offer the ability to embed live instances of classes within notebook pages, but these live examples are neither integrated with unit testing frameworks, not do they support custom tooling with the help of moldable tools to tailor the user experience.

% ============================================================
\section{Conclusion}\label{sec:conclusion}

Having tests be factories for examples is a tiny, but potentially groundbreaking enhancement.
EDD enhances TDD by making examples rather than tests be the focus of the development process.
By enhancing examples with lightweight, custom tools, they help users answer all kinds of analysis questions.
By embedding examples in live documentation, they make software systems explainable.

%\begin{acks}
%\end{acks}

% ============================================================

\bibliographystyle{ACM-Reference-Format}
\bibliography{eddBridge}

\end{document}
\endinput

% ============================================================

\begin{inparaenum}[(i)]
	\item 
\end{inparaenum}


\begin{figure}[h]
  \includegraphics[width=\columnwidth]{xxx}
  \caption{xxx.}
  \label{fig:xxx}
\end{figure}


